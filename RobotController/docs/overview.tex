\documentclass[a4paper,10pt]{article}
\usepackage[utf8]{inputenc}
\usepackage{amsmath}

% Title Page
\title{Overview of the Brain for the robot}
\author{}


\begin{document}
\maketitle

\tableofcontents

\section{Robot Brain}
The robot brain is the main class managing all the components of the robot.
The main cycle of the robot is inside this class.

\section{Learning Algorithms}
There is a general interface that is for every learning algorithm. It consists 
of \verb|next_evaluation()| function and is returning a new controller that 
needs to be swaped into \emph{HAL}

\subsection{RLPowerAlgorithm}
The \emph{RLPowerAlgorithm} is an implmenetation of the Learning Algorithm. It
stands for \emph{\textbf{R}einfoced \textbf{L}earning algorithm, 
\textbf{Po}licy learning by \textbf{W}eighting \textbf{E}xploration with the 
\textbf{R}eturn Algorithm}.

%TODO explaining how this algorithm works

\subsubsection{Example}
In the example we have a ranking list composed by:

\begin{center}
\begin{tabular}[c]{r *{5}{c}}
 Parameters & $P_1$ & $P_2$ & $P_3$ & $P_4$ & $P_5$ \\
 Fitness    & $10$  & $7$   & $6$   & $5$   & $2$   \\
 \\
 \multicolumn{3}{r}{Total} & \multicolumn{3}{l}{$30$} \\
%  \hline
\end{tabular}
\end{center}

If for $P_c$ we intend the current parameters and for $P_{c+1}$ we intend the
next evaluation. To find the next evaluation, this is the formula:

\begin{multline}
 P_{c+1} = P_c + \mathcal{N}\left(0, \sqrt[]{variance}\right) 
   + \frac{(P_1 - P_c) \cdot 10}{30} +\\
   + \frac{(P_2 - P_c) \cdot 7 }{30}
   + \frac{(P_3 - P_c) \cdot 6 }{30}
   + \frac{(P_4 - P_c) \cdot 5 }{30}
   + \frac{(P_5 - P_c) \cdot 2 }{30}
\end{multline}

\subsubsection{Ideas}
Here are some of the ideas for future improvements:
\begin{itemize}
 \item It would be nice to have a decade rate for the elements in the rankings,
 so that the old good walking patterns at some point they get replaced by new 
 ones. This is good in the event of an environment change (e.g. hitting the 
 wall) in which case all the previous walking patterns are probably going to be
 useless.
 \item A good idea is to have an automatic way of suddenly increase the mutation
 rate in case of sudden environment change (e.g. hitting the wall). This should
 help escape bad situations.
\end{itemize}


\section{Controllers}
A controller object is interpreting some inputs and is giving some outputs. The
object should be generated from the learning algorithm.

\subsection{RLPowerController}
TODO

\section{HAL}
\emph{HAL} stands for Hardware Abstraction Layer. Is the module responsible for
giving easy access to both inputs and outputs.

\subsection{Outputs}
\subsubsection{Servos}
This is taking the control input for the servo and sending it to the hardware.

\subsection{Inputs}
\subsubsection{PositionAsker}
\emph{PositionAsker} is a class that asks for the position from a remote server
which could be implemented with QRCode tracking or a more sophisticated system.

\end{document}
